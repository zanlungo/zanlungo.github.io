\documentclass[helvetica,narrow,openbib,notitle,noflag, nologo]{europecv}
\usepackage[a4paper,top=1.27cm,left=1cm,right=1cm,bottom=2cm]{geometry}
\usepackage[english]{babel} 
\usepackage{CJK}



%\usepackage{url}
%\usepackage[T1]{fontenc}
%\usepackage{ifpdf}
%\usepackage{bibentry}
\usepackage{hyperref}
\usepackage{enumerate}
\newcounter{papercount}
\newcounter{patentcount}

\ifpdf
    \usepackage[pdftex]{graphicx}
\else
     \usepackage{graphicx}
\fi

\renewcommand{\ttdefault}{phv} % Uses Helvetica instead of fixed width font

\newcommand*\ecvlanguagefooterjapanese[2][0pt]{\ecvitem[#1]{}{\quad\footnotesize{${}^{\mbox{\tiny#2}}$\textit{Holder of first (highest) level of Japanese proficiency, approved in 2008}}}}


\ecvlastname{Zanlungo}
\ecvfirstname{Francesco}
\ecvaddress{700-0927, Okayama-ken, Okayama-shi, Nishi Furu Matsu, 2-11-6, Japan}
\ecvtelephone{(+81) 0774-95-1561, (+81) 080-4018-2731}
\ecvemail{{zanlungo@atr.jp, francesco.zanlungo@gmail.com}}
\ecvhomepage{\url{www.irc.atr.jp/~zanlungo/}}
\ecvnationality{Italian, {\bf Holder of Japanese permanent residence permit}}
\ecvdateofbirth{10/03/1976}
\ecvgender{Male}

\ecvbeforepicture{\ecvspace{0.5cm}\raggedright}
%\ecvpicture[width=3cm]{francesco.eps}
\ecvpicture[width=3cm]{jappic.eps}
%\ecvpicture[width=3cm]{pic.eps}


\ecvafterpicture{\ecvspace{0.5cm}}
%\ecvpicture[width=1cm]{zeynep.jpg}



\begin{document}
\pagestyle{empty}
%\selectlanguage{english}
\ecvfootnote{For more information go to\\
\url{www.irc.atr.jp/~zanlungo/}}

\begin{europecv}
\ecvpersonalinfo[20pt]

\ecvsection{Research keywords}

\ecvitem{}{Complex Systems Modelling, Crowd Behaviour, Simulations, Robotics}

\ecvsection{Professional experience}

\ecvitem{ October 2022 - Present}{Contract consultant}
\ecvitem{}{Standard AI,  San Francisco, US}
\ecvitem{}{}

\ecvitem{ October 2021 - Present}{Contract lecturer}
\ecvitem{}{Okayama University,  Okayama, Japan}
\ecvitem{}{}

\ecvitem{ April 2021 - Present}{Lecturer}
\ecvitem{}{International Professional University of Technology, Osaka, Japan}
\ecvitem{}{Permanent position}
\ecvitem{}{}

\ecvitem{ April 2020 - Present}{Part time researcher}
\ecvitem{}{Okayama University,  Okayama, Japan}
\ecvitem{}{Studying the behaviour of pedestrians and robot-pedestrian interactions}
\ecvitem{}{}

\ecvitem{April 2017-March 2020}{Collaborative researcher}
\ecvitem{June 2020-Present}{Intelligent Robotics and Communication Laboratories, ATR, Kyoto, Japan}
\ecvitem{}{Studying the behaviour of pedestrians and robot-pedestrian interactions}
\ecvitem{}{}

\ecvitem{November 2016-March 2017}{Researcher}
\ecvitem{}{Intelligent Robotics and Communication Laboratories, ATR, Kyoto, Japan}
\ecvitem{}{Studying the behaviour of pedestrians and robot-pedestrian interactions}
\ecvitem{}{}
\newpage
\ecvitem{2015-2016}{Lecturer in Applied Mathematics}
\ecvitem{}{Kingston University, London, UK}
\ecvitem{}{Faculty of Science, Engineering and Computing, School of Computer Science and Mathematics}
\ecvitem{}{{\bf Tenured position}, resigned to go back to Japan for family related reasons}
\ecvitem{}{}

\ecvitem{2009-2015}{Researcher}
\ecvitem{}{Intelligent Robotics and Communication Laboratories, ATR, Kyoto, Japan}
\ecvitem{}{Studying the behaviour of pedestrians and robot-pedestrian interactions}
\ecvitem{}{}

\ecvitem{November 2008 and September 2009} {Visiting researcher}
\ecvitem{} {CPT (Theoretical Physics Center), Marseilles, France}
\ecvitem{} {Collaboration with Prof.~Sandro~Vaienti}
\ecvitem{}{}

\ecvitem{2008} {Instructor}
\ecvitem{} {Milan Polytechnic University}
\ecvitem{}{Teaching Introductory course of Mathematics (``College Algebra'')}
\ecvitem{}{}

\ecvitem{2007-2009} {Post-doctoral researcher}
\ecvitem{}{University of Bologna}
\ecvitem{}{Analysis of the effect of random noise and numerical round-off on discrete maps}
\ecvitem{}{}

\ecvitem{June-September 2005} {Visiting researcher}
\ecvitem{} {Artificial Life Laboratory at Nagoya University}
\ecvitem{} {Collaboration with Prof.~Takaya~Arita}
\ecvitem{}{}

\ecvsection{Journal editing}
\ecvitem{From August 2018} {Area Editor}
\ecvitem{}{Simulation Modelling Practice and Theory, Elsevier}
\ecvitem{}{}

\ecvsection{Professional experience outside research}

\ecvitem{From 2017}{Instructor of conversational Italian language}
\ecvitem{}{Japan-Italy Society of Okayama}

\ecvsection{Education and training}

\ecvitem{2004-2007} {Ph.D.~course}
\ecvitem{Major}{Theoretical Physics}
\ecvitem{Institution}{Graduate school of Physics, University of Bologna, Italy}
\ecvitem{Graduation Thesis} {Microscopic Dynamics of Artificial Life Systems, supervised by Prof.~Giorgio Turchetti}
\ecvitem{}{}

\ecvitem{2003} {Japanese language education} 
\ecvitem{Institution}{Yamasa Language school, Okazaki-shi, Aichi-ken, Japan}
\ecvitem{}{}

\ecvitem{2002} {Italian Laurea in Physics}
\ecvitem{}{\it (The Italian ``Laurea'' is legally equivalent to a Master degree. To obtain the degree, the candidate was supposed to work on a one year 
Graduation Thesis project requiring original research.)}
\ecvitem{Major}{Theoretical Physics}
\ecvitem{Institution}{University of Milan, Italy}
\ecvitem{Graduation Thesis} {Studio numerico della cascata ultravioletta nel modello $\phi^4$ classico (in Italian), supervised by Prof.~Claudio Destri}
\ecvitem{}{}
\newpage

\ecvsection{Languages}

\ecvmothertongue[10pt]{Italian}
\ecvlanguageheader{(*)}
\ecvlanguage{English}{\ecvCTwo}{\ecvCTwo}{\ecvCTwo}{\ecvCTwo}{\ecvCTwo}
\ecvlanguage{Japanese(**)}{\ecvCOne}{\ecvCOne}{\ecvCOne}{\ecvCOne}{\ecvCOne}
\ecvlastlanguage{Spanish}{\ecvCTwo}{\ecvCTwo}{\ecvCOne}{\ecvCOne}{\ecvCOne}
\ecvlastlanguage{Portuguese}{\ecvCOne}{\ecvCOne}{\ecvBTwo}{\ecvBTwo}{\ecvBOne}
\ecvlastlanguage{French}{\ecvBTwo}{\ecvCOne}{\ecvBOne}{\ecvBOne}{\ecvATwo}
\ecvlastlanguage{Turkish}{\ecvATwo}{\ecvATwo}{\ecvATwo}{\ecvAOne}{\ecvATwo}
\ecvlastlanguage{Mandarin Chinese}{\ecvAOne}{\ecvATwo}{\ecvAOne}{\ecvAOne}{\ecvATwo}
\ecvlanguagefooter{(*)}
\ecvlanguagefooterjapanese{(**)}



%%%%%%%%%%%%%%%%%%%%%%%%%%%%%%%%%%%%%%%%%%%%%%%%%%%%%%%%%%%%%%%%%%%%%%%%

\ecvsection{External funding}

\ecvitem{2021}{Collaborative researcher in the JSPS Kiban-A 18H04121 project {\it Research and development for mobile HRI and its interaction design theory} (Principal investigator T. Kanda)}
\ecvitem{}{Granted by the Japan Society for the Promotion of Science}
\ecvitem{}{Budget 2M Japanese Yen}

\ecvitem{2016}{I was, along with two colleagues, part of the Kingston University team that prepared the proposal for the H2020 EU ``Monica'' project, to which 26 European universities, research centres, industries and public institutions participated. 
The project has been approved with a budget of 15 million euros, 1 million of them corresponding to the Kingston University unit.}

\ecvsection{Awards}
\ecvitem{2016}{Awarded a Kingston University Mres studentship (i.e., a fund for a Master student)}

\ecvsection{Experience in event organisation}

\ecvitem{2006-2009} {In-chief of the organising committee}
\ecvitem{}{The Italian workshop
  on Biophysics (Biophys'06-09), held annually in Arcidosso, Grosseto, Italy.}


%%%%%%%%%%%%%%%%%%%%%%%%%%%%%%%%%%%%%%%%%%%%%%%%%%%%%%%%%%%%%%%%%%%%%%%%

\ecvsection{Computer skills}

\ecvitem{}{C, C++, Fortran, Matlab, Mathematica}
\ecvitem{}{MS Office, HTML}
\ecvitem{}{Latex}
\ecvitem{}{}

\ecvsection{Additional information}

\ecvitem{\large Home page}{\url{www.irc.atr.jp/~zanlungo/}}
\ecvitem{}{}

\ecvitem{\large Driving licence(s)}{B (cars)}
\ecvitem{}{}

\ecvitem{\large Personal interests} {Foreign languages, swimming, running, cycling, basketball, traveling, music, digital photography, books in general, my family.}

%%%%%%%%%%%%%%%%%%%%%%%%%%%%%%%%%%%%%%%%%%%%%%%%%%%%%%%%%%%%%%%%%%%%%%%%

\newpage

\ecvsection{Teaching experience}
\ecvitem{From 2021 }{Contract Lecturer at Okayama University, Okayama, Japan}
\ecvitem{} {\it Teaching ``Mechanics'' (in Japanese)}


\ecvitem{From 2021 }{Lecturer at the International Professional University of Technology, Osaka, Japan}
\ecvitem{} {\it Teaching courses in ``Linear Algebra'' (in Japanese, starting from 2021/9), ``Probability and Statistics'' (in Japanese, starting from 2022/4) and Scientific English (starting from 2023/4)}

\ecvitem{In 2018}{Contract Lecturer at Okayama University, Okayama, Japan}
\ecvitem{} {\it Teaching ``Global Studies II''}

\ecvitem{2016-2017 (Appointed, and prepared lecture notes, before resigning)}{Applications of Calculus, Partial Differential Equations module, School of Computer Science and Mathematics, Kingston University}
\ecvitem{} {\it This undergraduate course introduced the theory of Linear Partial Differential Equations. The course started with an introduction to the geometrical 
meaning of vector calculus leading to the expression of the Laplace operator in the principal curvilinear coordinate systems. Then the Heat, Wave, Poisson and Schr\"odinger equations were introduced, along with separation of variables 
solutions in Cartesian and spherical coordinates.}
\ecvitem{Course notes}{\url{https://www.dropbox.com/s/ja2arlaweqycn8b/notes_prova.pdf?dl=0}\\}

\ecvitem{2015-2016}{Mathematical and Numerical Methods, Numerical Linear Algebra module, School of Computer Science and Mathematics, Kingston University}
\ecvitem{} {\it This undergraduate course revised the main theoretical concepts of linear algebra (linear systems, vector space, linear operators, vector and matrix norm,
contraction theorem, eigenvalues and eigenvectors, matrix diagonalisation), and introduced numerical algorithms for the solution of related problems (Gaussian Elimination,
LU decomposition, iterative methods, eigenvalue power method).\\}
\ecvitem{Course notes}{\url{https://www.dropbox.com/s/z7d28niwldy734m/notes.pdf?dl=0}\\}

\ecvitem{2015-2016}{Mathematical Models and Computation, Programming module, School of Computer Science and Mathematics, Kingston University}
\ecvitem{} {\it This undergraduate course introduced the fundamental sorting and search algorithms, along with the theoretical concepts necessary for
their analysis (algorithm complexity). Part of the course was directed to practical exercitations aimed at acquiring the abilities for performing scientific programming.\\}

\ecvitem{2015-2016}{Engineering Mathematics and Computing, School of Civil Engineering, Kingston University}
\ecvitem{} {\it This undergraduate course revised the fundamental concepts of applied calculus (up to ordinary differential equations) and 
taught how to solve the related problems by  Matlab.\\}

\ecvitem{2008-2009} {Analytical Mechanics, Instructed by Prof.~Turchetti and F.~Zanlungo, Dep.~of Physics, Bologna University}
\ecvitem{} {\it This course was focused on a throughout analysis of Lagrangian Mechanics (Lagrange equations, symmetries,
central field, two body problem, stability, small oscillations, rigid body) and a solid introduction to Hamiltonian dynamics (Hamilton equations, canonical  transformations, Noether theorem,
integrable systems, Liouville theorem, ergodicity). My task on the course was to give part of the theoretical classes.\\}

\ecvitem{2008-2009} {Teaching assistant of the Institutions of Mathematics course, Instructed by Prof.~Bazzani, Milan Polytechnic University}
\ecvitem{} {\it An introductory calculus course, focused in particular on the concepts of real numbers, functions, limits and derivation,
along with some notions of linear algebra. My task on the course was to give some theoretical classes in absence of Prof. Bazzani, hold practice sessions and prepare examination tests.\\}

\ecvitem{2008} {Introductory course of Mathematics (``College Algebra''), Milan Polytechnic University}
\ecvitem{} {\it A course intended for those students that passed the University entry exam but scored poorly in
mathematics, focused mainly on the concept of elementary real functions.\\}

\ecvitem{2007-2008} {Teaching assistant of the Numerical Methods course, Instructed by Profs.~Turchetti and Bazzani, Master course in Physics, Bologna University}
\ecvitem{} {\it This course was focused on an introduction of 
numerical methods for physical sciences (interpolation, numerical solution of non linear
equations,
numerical integration, numerical solution of differential equations, stochastic systems). My task on the course was to give a few theoretical classes
and to assist students during practice sessions.\\}
\ecvitem{2007-2008}{Teaching assistant of the Complex Systems Laboratory course, Instructed by Dr.~Giorgini, Dep.~of Physics, Bologna University}
\ecvitem{} {\it For this course, I prepared lectures on genetic algorithms, population dynamics (evolutionary game theory) and neural networks.\\}

\ecvsection{Research experience}

\ecvitem{Crowd dynamics}{\it Mathematical modelling of pedestrian behaviour, crowd dynamics and group behaviour, in collaboration with T.~Kanda}
\ecvitem{}{In ATR we collected a large amount of data concerning the behaviour of pedestrians 
in experimental settings and in real world environments, which I used to develop original models of pedestrian and crowd dynamics. More in detail, the major findings regarded:
\begin{enumerate}
\item The need to include a velocity dependent potential in a collision avoiding model for pedestrians, and the development of a corresponding mathematical and computational model {\bf[12]}.
\item The improvement of the above model by taking in account the asymmetrical shape of human bodies {\bf[19]}, {\it (work based on data provided by the Nishinari lab. of Tokyo University)}.
\item The development of ``Congestion Number'', a mathematical tool to asses the state of a human pedestrian crowd {\bf[20]}, {\it (in collaboration with the Nishinari lab. of Tokyo University)}
\item The tendency of (Japanese) pedestrians to walk on the left side of corridors, and to overtake other pedestrians on the right side, 
and the development of a method to introduce in a realistic way such a tendency in any pedestrian collision avoidance model {\bf[11,38]}.
\item Large pedestrian groups are not stable, and usually break up in more stable 2 or 3 pedestrian sub-units {\bf[36]}.
\item A mathematical model for the behaviour of social pedestrian groups, which was able to correctly predict the shape and velocity of pedestrian groups in low density,
large environments {\bf[9]}.   
\item Empirical study and mathematical modelling of how crowd density and other environment features affect the behaviour of pedestrians and in particular of groups {\bf[6,7,34,35]}. 
\item How group composition and social roles affect the behaviour of pedestrian groups, and how this information may be used to automatically recognise groups and their composition  {\bf[1,2,5,23,26,27,33]}. 
\item How gestures affect the behaviour of pedestrian groups {\bf[4,29]}.
\item How the presence of groups affects crowd dynamics {\bf[24]} 
\end{enumerate}
\\
}

\ecvitem{Human-Robot interaction}{\it Socially acceptable mobile robot navigation, in collaboration with T.~Kanda}
\ecvitem{}{While working at  ATR I have been also involved in more engineering oriented works, such as the development of a robot able to smoothly navigate inside a human crowd {\bf[8,22,28,30,39,41]},
and the development of algorithms to automatically detect pedestrian walking goals {\bf[37]} and pedestrian groups {\bf[10,31,40]}.
\\
}
\ecvitem{Discrete chaotic systems} {\it Analysis of the effect of noise on discrete maps, in collaboration with S.~Vaienti and G.~Turchetti}
 \ecvitem{}{Development of a method
 to find a threshold beyond which the numerical results on chaotic maps are not reliable, and
analysis of the differences between the effect of random noise and the effect of numerical round-off on the dynamics of the map {\bf[13,14,15]}.\\}
\newpage
\ecvitem{Evolutionary dynamics of agent systems} {\it Microscopic Dynamics of Artificial Life Systems  (Ph.D.~thesis, sup.~G.~Turchetti, in collaboration with T.~Arita)}
\ecvitem{}{Using an approach combining cellular automata or agent models with differential equation (replicator dynamics) models, I studied:
\begin{enumerate}
\item The Immune System T cell clonal expansion {\bf[21]}.
\item The relation between the evolution of collision avoidance strategies and the evolution of
a {\it Theory of Mind} {\bf[18,43,44]}.
\item The evolution of ``traffic conventions'' (such as driving on the left or right side of streets) in a mobility system
{\bf[16,45]}.
\item The consequences of the fact that interactions dependent on vision (such as the collision avoidance in crowd dynamics)
do not follow the action-reaction law of dynamics {\bf[17,25]}.
\end{enumerate}\\}

\ecvitem{Numerical study of statistical properties of relativistic fields} {\it Numerical study of the ultraviolet cascade in  $\phi^4$ classical model (Master thesis, sup.~C.~Destri)}
\ecvitem{}{Using a numerical algorithm that treats time and space in a symmetrical way, preserving thus the relativistic structure of the field theory, and
conserving energy at machine precision, I studied the energy diffusion to the higher (ultraviolet) modes of a relativistic scalar field with a quartic interaction term. 
The results were compared with a more traditional numerical treatment of hyperbolic partial differential equations.\\}
\newpage

\ecvsection{Publications}

\ecvitem[10pt]{\large Journal papers and book chapters}{}

\ecvitem{}{F. Zanlungo, C. Feliciani, Z. Y\"ucel, K. Nishinari, T. Kanda}
\ecvitem{}{Macroscopic and microscopic dynamics of a pedestrian cross-flow: Part I, experimental analysis}
\ecvitem{}{Safety Science (in press).}
\ecvitem{}{}

\stepcounter{papercount}
\ecvitem{\thepapercount}{F.~Zanlungo, Z. Y{\"u}cel, T. Kanda}
\ecvitem{}{\it Intrinsic group behaviour II: On the dependence of triad spatial dynamics on social and personal features; and on the effect of social interaction on small group dynamics}
\ecvitem{}{PloS One, Vol 14, No 12, pp e0225704, 2019}
\ecvitem{}{doi: 10.1371/journal.pone.0225704 {\bf (impact factor 2.776)}}
\ecvitem{}{}

\stepcounter{papercount}
\ecvitem{\thepapercount}{Z. Y\"ucel, F. Zanlungo, C. Feliciani, Claudio, A. Gregorj,  T. Kanda}
\ecvitem{}{\it Identification of social relation within pedestrian dyads}
\ecvitem{}{PloS One, Vol 14, No 10, pp e0223656, 2019}
\ecvitem{}{doi: 10.1371/journal.pone.0223656 {\bf (impact factor 2.776)}}
\ecvitem{}{}

\stepcounter{papercount}
\ecvitem{\thepapercount}{39 authors including F.~Zanlungo}
\ecvitem{}{\it A Glossary for Research on Human Crowd Dynamics}
\ecvitem{}{Collective Dynamics, Vol. 4, pp. 1-13, 2019}
\ecvitem{}{doi: 10.17815/CD.2019.19}
\ecvitem{}{}

\stepcounter{papercount}
\ecvitem{\thepapercount}{Z.~Y\"ucel, F.~Zanlungo and M.~Shiomi}
\ecvitem{}{\it Modeling the impact of interaction on pedestrian group motion}
\ecvitem{}{Advanced Robotics, Vol. 32, No 3, pp. 137-147, 2018 {\bf (impact factor 0.92)}}
\ecvitem{}{doi: 10.1080/01691864.2017.1421481}
\ecvitem{}{}

\stepcounter{papercount}
\ecvitem{\thepapercount}{F.~Zanlungo, Z.~Y\"ucel, D.~Br\v{s}\v{c}i\'c, T.~Kanda, N.~Hagita} 
\ecvitem{}{\it Intrinsic group behaviour: dependence of pedestrian dyad dynamics on principal social and personal features}
\ecvitem{}{Plos One 0187253, 2017 {\bf (impact factor 3.54)}}
\ecvitem{}{doi: 10.1371/journal.pone.0187253}
\ecvitem{}{}

\stepcounter{papercount}
\ecvitem{\thepapercount}{F.~Zanlungo, T.~Kanda} 
\ecvitem{}{\it A mesoscopic model for the effect of density on pedestrian group dynamics}
\ecvitem{}{Europhysics Letters, Vol. 111, No 3, pp. 38007, 2015 {\bf (impact factor 2.095)}}
\ecvitem{}{doi: 10.1209/0295-5075/111/38007}
\ecvitem{}{}

\stepcounter{papercount}
\ecvitem{\thepapercount}{F.~Zanlungo, D.~Br\v{s}\v{c}i\'c, T.~Kanda} 
\ecvitem{}{\it Spatial-size scaling of pedestrian groups under growing density conditions}
\ecvitem{}{Physical Review E Vol. 91 No 6, pp. 062810, 2015 {\bf (impact factor 2.288)}}
\ecvitem{}{doi: 10.1103/PhysRevE.91.062810}
\ecvitem{}{}

\stepcounter{papercount}
\ecvitem{\thepapercount}{M.~Shiomi, F.~Zanlungo, K.~Hayashi , T.~Kanda}
\ecvitem{}{\it Towards a Socially Acceptable Collision Avoidance for a Mobile Robot Navigating Among Pedestrians Using a Pedestrian Model}
\ecvitem{}{International Journal of Social Robotics, Vol. 6, No 3, pp 443-455, 2014 {\bf (impact factor 1.207)}}
\ecvitem{}{doi: 10.1007/s12369-014-0238-y}
\ecvitem{}{}

\stepcounter{papercount}
\ecvitem{\thepapercount}{F.~Zanlungo, T.~Ikeda, T.~Kanda} 
\ecvitem{}{\it Potential for the dynamics of pedestrians in a socially interacting group}
\ecvitem{}{Physical Review E Vol. 89, No 1, pp. 012811, 2014 {\bf (impact factor 2.288)}}
\ecvitem{}{doi: 10.1103/PhysRevE.89.012811}
\ecvitem{}{\bf (Paper chosen as ``editor suggestion'', i.e.~as being of particular clarity and importance)}
\ecvitem{}{}

\stepcounter{papercount}
\ecvitem{\thepapercount}{Z.~Y\"ucel, F.~Zanlungo, T.~Ikeda, T.~Miyashita, N.~Hagita}
\ecvitem{}{\it Deciphering the crowd: Modeling and identification of pedestrian group motion}
\ecvitem{}{Sensors, Vol. 13, No. 1, pp. 875-897, 2013 
{\bf (impact factor 1.953)}}
\ecvitem{}{doi: 10.3390/s130100875}
\ecvitem{}{}

\stepcounter{papercount}
\ecvitem{\thepapercount}{F.~Zanlungo, T.~Ikeda, T.~Kanda}
\ecvitem{}{\it A microscopic social norm model to obtain realistic macroscopic velocity and density pedestrian distributions}
\ecvitem{}{PLoS ONE Vol. 7, No 12, pp. e50720, 2012 {\bf (impact factor 3.73)}}
\ecvitem{}{doi: 10.1371/journal.pone.0050720}
\ecvitem{}{}

\stepcounter{papercount}
\ecvitem{\thepapercount}{F.~Zanlungo, T.~Ikeda, T.~Kanda}
\ecvitem{}{\it Social force model with explicit collision prediction}
\ecvitem{}{Europhysics Letters, Vol. 93, No. 6, pp. 68005, 2011 {\bf (impact factor 2.171)}}
\ecvitem{}{doi: 10.1209/0295-5075/93/68005}
\ecvitem{}{}

\stepcounter{papercount}
\ecvitem{\thepapercount}{G.~Turchetti, S.~Vaienti and F.~Zanlungo}
\ecvitem{}{\it Asymptotic distribution of global errors in the numerical computations of dynamical systems}
\ecvitem{}{Physica A, Vol. 389, No 21, pp. 4994-5006, 2010 {\bf (impact factor 1.521)}}
\ecvitem{}{doi: 10.1016/j.physa.2010.06.060}
\ecvitem{}{}

\stepcounter{papercount}
\ecvitem{\thepapercount}{G.~Turchetti, S.~Vaienti and F.~Zanlungo}
\ecvitem{}{\it Relaxation to the asymptotic distribution of global errors due to round off}
\ecvitem{}{Europhysics Letters, Vol. 89, No 4, pp. 40006, 2010 {\bf (impact factor 2.753)}}
\ecvitem{}{doi: 10.1209/0295-5075/89/40006}
\ecvitem{}{}

\stepcounter{papercount}
\ecvitem{\thepapercount}{P.~Marie, G.~Turchetti, S.~Vaienti and F.~Zanlungo}
\ecvitem{}{\it Error distribution in randomly perturbed orbits}
\ecvitem{}{Chaos: An Interdisciplinary Journal of Nonlinear Science, Vol. 19, No 4, pp. 043118, 2009 {\bf (impact factor 1.795)}}
\ecvitem{}{doi: 10.1063/1.3267510}
\ecvitem{}{}

\stepcounter{papercount}
\ecvitem{\thepapercount}{F.~Zanlungo, T.~Arita, S.~Rambaldi}
\ecvitem{}{\it Emergence of a traffic flow convention in a multiagent model}
\ecvitem{}{Advances in Complex Systems. Vol. 11, No 5, pp. 789-802, 2008}
\ecvitem{}{doi: 10.1142/S0219525908001921}
\ecvitem{}{}


\stepcounter{papercount}
\ecvitem{\thepapercount}{G.~Turchetti, F.~Zanlungo, B.~Giorgini}
\ecvitem{}{\it Dynamics and thermodynamics of a gas of automata}
\ecvitem{}{Europhysics Letters, Vol. 78, No 5, pp. 58003, 2007 \bf {(impact factor 2.206)}}
\ecvitem{}{doi: 10.1209/0295-5075/78/58003}
\ecvitem{}{}

\stepcounter{papercount}
\ecvitem{\thepapercount}{F.~Zanlungo}
\ecvitem{}{\it A collision avoiding mechanism based on a theory of mind}
\ecvitem{}{Advances in Complex Systems. Vol.~10 suppl. No.~2, pp.~363-371, 2007}
\ecvitem{}{doi: 10.1142/S0219525907001410}
\ecvitem{}{}

\ecvitem[10pt]{\large Under review/ In preparation}

\stepcounter{papercount}
\ecvitem{\thepapercount}{F. Zanlungo, C. Feliciani, Z. Y\"ucel, K. Nishinari, T. Kanda}
\ecvitem{}{\it Analysis and modelling of macroscopic and microscopic dynamics of a pedestrian cross-flow}
\ecvitem{}{arXiv:2112.12304}
\ecvitem{}{}

\newpage

\stepcounter{papercount}
\ecvitem{\thepapercount}{F. Zanlungo, C. Feliciani, Z. Y\"ucel, K. Nishinari, T. Kanda}
\ecvitem{}{\it Some considerations on crowd Congestion Level}
\ecvitem{}{arXiv:2004.01883}
\ecvitem{}{}

\ecvitem[10pt]{\large Book chapters}


\stepcounter{papercount}
\ecvitem{\thepapercount}{F.~Zanlungo, G.~Turchetti, S.~Rambaldi}
\ecvitem{}{\it An Automata Based Microscopic Model Inspired by Clonal Expansion}
\ecvitem{}{Mathematical Modeling of Biological Systems, Volume II. A.~Deutsch et al.~(eds.), Birkh{\"a}user, Boston, pp.~133-144, 2008}
\ecvitem{}{doi: 10.1007/978-0-8176-4556-4\_12}
\ecvitem{}{}


\ecvitem[10pt]{\large Peer-reviewed conference papers}

\stepcounter{papercount}
\ecvitem{\thepapercount}{E.~Repiso, F.~Zanlungo, T.~Kanda, A.~Garrell, A.~Sanfeliu}
\ecvitem{} {\it People's V-Formation and Side-by-Side Model Adapted to Accompany
Groups of People by Social Robots}
\ecvitem{}{International Conference on Intelligent Robots
and Systems 2019, pp. 2082-2088}
\ecvitem{}{Nov 4-8 2019, Macau, China}
\ecvitem{}{doi: 10.1109/IROS40897.2019.8968601}
\ecvitem{}{}

\stepcounter{papercount}
\ecvitem{\thepapercount}{Z.~Y\"ucel, F.~Zanlungo, T.~Kanda}
\ecvitem{} {\it Gender profiling of pedestrian dyads}
\ecvitem{}{Traffic and Granular Flow Conference 2019, pp. 299-305}
\ecvitem{}{July 2-5 2019, Pamplona, Spain}
\ecvitem{}{doi: 10.1007/978-3-030-55973-1\_37}
\ecvitem{}{}

\stepcounter{papercount}
\ecvitem{\thepapercount}{F.~Zanlungo, L.~Crociani, Z.~Y\"ucel, T.~Kanda}
\ecvitem{} {\it The effect of social groups on the dynamics of bi-directional pedestrian flow: a numerical study}
\ecvitem{}{Traffic and Granular Flow Conference 2019, pp. 307-313}
\ecvitem{}{July 2-5 2019, Pamplona, Spain}
\ecvitem{}{doi: 10.1007/978-3-030-55973-1\_38}
\ecvitem{}{}

\stepcounter{papercount}
\ecvitem{\thepapercount}{C.~Feliciani, F.~Zanlungo, K.~Nishinari,  T.~Kanda} 
\ecvitem{}{\it Thermodynamics of a gas of pedestrians: Theory and experiment}
\ecvitem{}{Pedestrian and Evacuation Conference 2018}
\ecvitem{}{Collective Dynamics, Vol 5, pp. 440-447, 2020}
\ecvitem{}{Aug 21-24 2018, Lund, Sweden}
\ecvitem{}{doi: 10.17815/CD.2020.97}
\ecvitem{}{}

\stepcounter{papercount}
\ecvitem{\thepapercount}{Z.~Y\"ucel, F.~Zanlungo, C.~Feliciani, T.~Kanda}
\ecvitem{} {\it Estimating social relation from trajectories}
\ecvitem{}{Pedestrian and Evacuation Conference 2018}
\ecvitem{}{Collective Dynamics, Vol 5, pp. 222-229, 2020}
\ecvitem{}{Aug 21-24 2018, Lund, Sweden}
\ecvitem{}{doi: 10.17815/CD.2020.54}
\ecvitem{}{}

\stepcounter{papercount}
\ecvitem{\thepapercount}{F.~Zanlungo, Z.~Y\"ucel, T.~Kanda}
\ecvitem{} {\it Social group behaviour of triads. Dependence on purpose and gender}
\ecvitem{}{Pedestrian and Evacuation Conference 2018}
\ecvitem{}{Collective Dynamics, Vol 5, pp. 118-125, 2020}
\ecvitem{}{Aug 21-24 2018, Lund, Sweden}
\ecvitem{}{doi: 10.17815/CD.2020.90}
\ecvitem{}{}

\stepcounter{papercount}
\ecvitem{\thepapercount}{F.~Zanlungo, Z.~Y\"ucel, F.~Ferreri, J.~Even, L.Y.~Morales Saiki, T.~Kanda}
\ecvitem{} {\it Pedestrian models for robot motion}
\ecvitem{}{Pedestrian and Evacuation Conference 2018}
\ecvitem{}{Collective Dynamics, Vol 5, pp. 525-527, 2020}
\ecvitem{}{Aug 21-24 2018, Lund, Sweden}
\ecvitem{}{doi: 10.17815/CD.2020.90}
\ecvitem{}{}



\stepcounter{papercount}
\ecvitem{\thepapercount}{Z.~Y\"ucel, F.~Zanlungo and M.~Shiomi}
\ecvitem{} {\it Walk the talk: Gestures in mobile interaction}
\ecvitem{}{International Conference on Social Robotics 2017, pp. 220-230}
\ecvitem{}{Nov 22-24 2017, Tsukuba, Japan}
\ecvitem{}{doi: 10.1007/978-3-319-70022-9\_22}
\ecvitem{}{}

\stepcounter{papercount}
\ecvitem{\thepapercount}{F.~Zanlungo, Z.~Y\"ucel, F.~Ferreri, J.~Even, L.Y.~Morales Saiki, T.~Kanda}
\ecvitem{} {\it Social group motion in robots}
\ecvitem{}{International Conference on Social Robotics 2017, pp. 474-484, Tsukuba, Japan}
\ecvitem{}{doi: 10.1007/978-3-319-70022-9\_47}
\ecvitem{}{}

\stepcounter{papercount}
\ecvitem{\thepapercount}{D.~Br\v{s}\v{c}i\'c, F.~Zanlungo, T.~Kanda}
\ecvitem{}{\it Modelling of Pedestrian groups and application to group recognition}
\ecvitem{}{40th International Convention on Information Information and Communication Technology, Electronics and Microelectronics (MIPRO), 2017, pp.~564-569, Opatija, Croatia}
\ecvitem{}{doi: 10.23919/MIPRO.2017.7973489}
\ecvitem{}{}

\stepcounter{papercount}
\ecvitem{\thepapercount}{K.~Kamei, F.~Zanlungo, T.~Kanda, Y.~Horikawa, T.~Miyashita, N.~Hagita}
\ecvitem{}{\it Cloud networked robotics for social robotic services extending robotic functional service standards to support autonomous mobility system in social environments}
\ecvitem{}{International Conference on Ubiquitous Robots and Ambient Intelligence (URAI), 2017, pp.~897-902, Jeju, South Korea}
\ecvitem{}{doi: 10.1109/URAI.2017.7992862}
\ecvitem{}{}

\stepcounter{papercount}
\ecvitem{\thepapercount}{F.~Zanlungo, Z.~Y\"ucel, T.~Kanda}
\ecvitem{}{\it The effect of social roles on group behaviour}
\ecvitem{}{Pedestrian and Evacuation Conference 2016, pp. 243-249, Hefei, China}
\ecvitem{}{doi: 10.17815/CD.2016.11}
\ecvitem{}{}



\stepcounter{papercount}
\ecvitem{\thepapercount}{F.~Zanlungo, D.~Br\v{s}\v{c}i\'c, T.~Kanda}
\ecvitem{}{\it Pedestrian group behaviour analysis under different density conditions}
\ecvitem{}{Pedestrian and Evacuation Conference 2014, Delft, Netherlands}
\ecvitem{}{Transportation Research Procedia Vol. 2, 149-158, 2014}
\ecvitem{}{doi: 10.1016/j.trpro.2014.09.020}
\ecvitem{}{}

\stepcounter{papercount}
\ecvitem{\thepapercount}{D.~Br\v{s}\v{c}i\'c, F.~Zanlungo, T.~Kanda}
\ecvitem{}{\it Density and velocity patterns during one year of pedestrian tracking}
\ecvitem{}{Pedestrian and Evacuation Conference 2014, Delft, Netherlands}
\ecvitem{}{Transportation Research Procedia 2, 77-86, 2014}
\ecvitem{}{doi: 10.1016/j.trpro.2014.09.011}
\ecvitem{}{}

\stepcounter{papercount}
\ecvitem{\thepapercount}{F.~Zanlungo, T.~Kanda}
\ecvitem{} {\it Do walking pedestrians stabily interact inside a large group? Analysis of group and sub-group spatial structure}
\ecvitem{}{The annual meeting of cognitive science society (CogSci) 2013, Vol. 35, No. 35, pp. 3847-3852, Berlin, Germany}
\ecvitem{}{}

\stepcounter{papercount}
\ecvitem{\thepapercount}{T.~Ikeda, Y.~Chigodo, D.~Rea, F.~Zanlungo, M.~Shiomi, T.~Kanda}
\ecvitem{}{\it Modeling and Prediction of Pedestrian Behavior based on the Sub-goal Concept}
\ecvitem{}{Robotics: Science and Systems (RSS) 2013, pp. 137-144, Sidney, Australia (acceptance rate 33\%)}
\ecvitem{}{doi: 10.15607/RSS.2012.VIII.018}
\ecvitem{}{}

\newpage

\stepcounter{papercount}
\ecvitem{\thepapercount}{F.~Zanlungo, Y.~Chigodo, T.~Ikeda, T.~Kanda}
\ecvitem{}{\it Experimental study and modelling of pedestrian space occupation and motion pattern in a real world environment}
\ecvitem{}{Pedestrian and Evacuation Dynamics 2012, Zurich, Switzerland}
\ecvitem{}{Weidmann et al.~(eds.), pp.~289-304, Springer, (published as a book in 2014)}
\ecvitem{}{doi: 10.1007/978-3-319-02447-9\_24}
\ecvitem{}{}

\stepcounter{papercount}
\ecvitem{\thepapercount}{M.~Shiomi, F.~Zanlungo, K.~Hayashi, T.~Kanda}
\ecvitem{}{\it A Framework with a Pedestrian Simulator for Deploying Robots into a Real Environment}
\ecvitem{}{International Conference on Simulation, Modeling, and Programming for Autonomous Robots 2012 (SIMPAR), pp. 185-196, (acceptance rate 35\%)}
\ecvitem{}{doi: 10.1007/978-3-642-34327-8\_19}
\ecvitem{}{}

\stepcounter{papercount}
\ecvitem{\thepapercount}{Z.~Y\"ucel, F.~Zanlungo, T.~Ikeda, T.~Miyashita, N.~Hagita}
\ecvitem{}{\it Modeling Indicators of Coherent Motion}
\ecvitem{}{International Conference on Intelligent Robots and Systems (IROS) 2012, pp 2134--2140, Algarve, Portugal (acceptance rate 39\%) 2012}
\ecvitem{}{doi: 10.1109/IROS.2012.6385744}
\ecvitem{}{}

\stepcounter{papercount}
\ecvitem{\thepapercount}{M.~D.~Cooney, F.~Zanlungo, S.~Nishio, H.~Ishiguro}
\ecvitem{}{\it Designing a Flying Humanoid Robot (FHR): Effects of Flight on Interactive Communication}
\ecvitem{}{International Symposium on Robot and Human Interactive Communication (IEEE RO-MAN) 2012, pp. 364-371, 2012, Paris, France}
\ecvitem{}{doi: 10.1109/ROMAN.2012.6343780}
\ecvitem{}{}

\stepcounter{papercount}
\ecvitem{\thepapercount}{A.~Bazzani, B.~Giorgini, F.~Zanlungo and S.~Rambaldi}
\ecvitem{}{\it Cognitive Dynamics in an automata gas}
\ecvitem{}{Artificial Life and Evolutionary Computation, pp. 3-19, Wivace 2008, Venice, Italy}
\ecvitem{}{doi: 10.1142/9789814287456\_0001}
\ecvitem{}{}

\stepcounter{papercount}
\ecvitem{\thepapercount}{F.~Zanlungo}
\ecvitem{} {\it Evolution of high level recursive thinking in a collision avoiding agent model}
\ecvitem{}{Artificial Life and Evolutionary Computation, pp. 155-164, Wivace 2008, Venice, Italy}
\ecvitem{}{doi: 10.1142/9789814287456\_0014}
\ecvitem{}{}

\stepcounter{papercount}
\ecvitem{\thepapercount}{F.~Zanlungo, A.~Bazzani, B.~Giorgini, S.~Rambaldi, G.~Servizi and G.~Turchetti}
\ecvitem{} {\it An evolutionary crowd dynamics model}
\ecvitem{}{European Conference on Complex Systems 2007, Dresden Germany}
\ecvitem{}{}

\stepcounter{papercount}
\ecvitem{\thepapercount}{F.~Zanlungo, T.~Arita}
\ecvitem{} {\it Evolutionary Simulation of an Agent Based Mobility System Using Indirect Communication}
\ecvitem{}{International Symposium of Artificial Life and Robotics (A-Life) 2006, pp.~319-322, Oita, Japan}
\ecvitem{}{}



\ecvitem[10pt]{\large Other presentations at conferences}

\stepcounter{papercount}
\ecvitem{\thepapercount}{F.~Zanlungo, Z.~Y\"ucel, F.~Ferreri, J.~Even, L.Y.~Morales Saiki, T.~Kanda}
\ecvitem{} {\it Autonomous vehicles moving as a human group}
\ecvitem{}{Poster presentation at IROS 2017}
\ecvitem{}{}

\stepcounter{papercount}
\ecvitem{\thepapercount}{F.~Zanlungo, G.~Turchetti}
\ecvitem{}{\it Dynamics and Thermodynamics of Automata with a visual cone. Comparison with a recursive thinking model}
\ecvitem{}{Dynamics and Thermodynamics of Systems with Long Range Interactions: Theory and Experiments, 2007}
\ecvitem{}{}

\stepcounter{papercount}
\ecvitem{\thepapercount}{F.~Zanlungo, G.~Turchetti}
\ecvitem{}{\it An evolutionary collision avoiding model based on the theory of mind}
\ecvitem{}{International Conference on the Simulation of adaptive behavior (SAB) 2006, Rome, Italy}
\ecvitem{}{}

\stepcounter{papercount}
\ecvitem{\thepapercount}{F.~Zanlungo, G.~Turchetti}
\ecvitem{}{\it Dynamics and thermodynamics of a gas of automata}
\ecvitem{}{Italian Workshop in Artificial Life (WIVA3), 2006}
\ecvitem{}{}

\stepcounter{papercount}
\ecvitem{\thepapercount}{G.~Turchetti, F.~Zanlungo}
\ecvitem{}{\it Termodinamica di un gas di automi (in Italian)}
\ecvitem{}{Italian Workshop in Artificial Life (WIVA2), 2005, Rome, Italy}
\ecvitem{}{}

\stepcounter{papercount}
\ecvitem{\thepapercount}{G.~Turchetti, S.~Rambaldi, G.~Salustri and F.~Zanlungo}
\ecvitem{}{\it Mathematical models of clonal expansion}
\ecvitem{}{WSEAS Transactions on Biology and Biomedicine 1, 373-378, 2004}
\ecvitem{}{}

\ecvitem[10pt]{\large Invited talks}

\stepcounter{papercount}
\ecvitem{\thepapercount}{\it Pedestrian models: current state and perspectives}
\ecvitem{}{Kyoto University}
\ecvitem{}{Kyoto, Japan, 2019}
\ecvitem{}{}

\stepcounter{papercount}
\ecvitem{\thepapercount}{\it Pedestrian group behaviour}
\ecvitem{}{Kyoto University}
\ecvitem{}{Kyoto, Japan, 2019}
\ecvitem{}{}

\stepcounter{papercount}
\ecvitem{\thepapercount}{\it Pedestrian group behaviour}
\ecvitem{}{Alicante University}
\ecvitem{}{Alicante, Spain, 2019}
\ecvitem{}{}

\stepcounter{papercount}
\ecvitem{\thepapercount}{\it Pedestrian group behaviour}
\ecvitem{}{Polythechnic University of Catalonia}
\ecvitem{}{Barcelona, Spain, 2019}
\ecvitem{}{}

\stepcounter{papercount}
\ecvitem{\thepapercount}{\it Pedestrian group behaviour}
\ecvitem{}{Symposium on Physics and Psychology of Human Crowd Dynamics}
\ecvitem{}{Leiden, Netherlands, 2018}
\ecvitem{}{}

\stepcounter{papercount}
\ecvitem{\thepapercount}{\it Pedestrian group behaviour}
\ecvitem{}{Department of Physics of Bologna University}
\ecvitem{}{Bologna, Italy, 2018}
\ecvitem{}{}

\stepcounter{papercount}
\ecvitem{\thepapercount}{\it Pedestrian group behaviour}
\ecvitem{}{Linnaeus University}
\ecvitem{}{V\"axj\"o, Sweden, 2018}
\ecvitem{}{}

\stepcounter{papercount}
\ecvitem{\thepapercount}{\it Pedestrian group behaviour}
\ecvitem{}{University of Milano Bicocca}
\ecvitem{}{Milan, Italy, 2017}
\ecvitem{}{}

\stepcounter{papercount}
\ecvitem{\thepapercount}{\it Pedestrian group behaviour}
\ecvitem{}{Tokyo University, Non-linear seminar, Nishinari Laboratory}
\ecvitem{}{Tokyo, Japan, 2016}
\ecvitem{}{}

\stepcounter{papercount}
\ecvitem{\thepapercount}{\it Potential for the dynamics of pedestrians in a socially interacting group}
\ecvitem{}{Department of Physics of Bologna University}
\ecvitem{}{Bologna, Italy, 2014}
\ecvitem{}{}

\stepcounter{papercount}
\ecvitem{\thepapercount}{\it Potential for the dynamics of pedestrians in a socially interacting group}
\ecvitem{}{Artificial Life Laboratory of Nagoya University (Arita Lab)}
\ecvitem{}{Nagoya, Japan, 2014}
\ecvitem{}{}

\stepcounter{papercount}
\ecvitem{\thepapercount}{\it Experimental study and modelisation of pedestrian space occupation and motion pattern in a real world environment}
\ecvitem{}{Department of Physics of Bologna University}
\ecvitem{}{Bologna, Italy, 2012}
\ecvitem{}{}

\stepcounter{papercount}
\ecvitem{\thepapercount}{\it Experimental study and modelisation of pedestrian space occupation and motion pattern in a real world environment}
\ecvitem{}{Artificial Life Laboratory of Nagoya University (Arita Lab)}
\ecvitem{}{Nagoya, Japan, 2012}
\ecvitem{}{}

\stepcounter{papercount}
\ecvitem{\thepapercount}{\it Social force model with explicit collision prediction}
\ecvitem{}{Artificial Life Laboratory of Nagoya University (Arita Lab)}
\ecvitem{}{Nagoya, Japan, 2011}
\ecvitem{}{}

\stepcounter{papercount}
\ecvitem{\thepapercount}{\it Evolution of Behaviours in Artificial Life}
\ecvitem{}{International Summer School: Interfacing Sciences and Humanities}
\ecvitem{}{Rimini, Italy, 2009}
\ecvitem{}{}

\stepcounter{papercount}
\ecvitem{\thepapercount}{\it Chaos and Complexity}
\ecvitem{}{International Summer School: Interfacing Sciences and Humanities}
\ecvitem{}{Rimini, Italy, 2009}
\ecvitem{}{}

\stepcounter{papercount}
\ecvitem{\thepapercount}{\it Error statistics in perturbed discrete dynamical systems}
\ecvitem{}{Department of Mathematics of Bologna University}
\ecvitem{}{Bologna, Italy, 2009}
\ecvitem{}{}

\stepcounter{papercount}
\ecvitem{\thepapercount}{\it Evolutionary techniques in a traffic model}
\ecvitem{}{Nagatani Laboratory of Shizuoka University}
\ecvitem{}{Hamamatsu, Japan, 2008}
\ecvitem{}{}

\newpage 
\ecvitem[10pt]{\large Japanese language papers}

\stepcounter{papercount}
\ecvitem{\thepapercount}{\begin{CJK}{UTF8}{min} 林宏太郎、塩見昌裕、Francesco ZANLUNGO、神田崇行 \end{CJK}}
\ecvitem{}{\begin{CJK}{UTF8}{min} 歩行者モデルを用いた話しかけやすい移動行動 \end{CJK}}
\ecvitem{}{\begin{CJK}{UTF8}{min} 日本ロボット学会第32回学術講演会講演論文集RJS2014, 3P2-07, 2014 \end{CJK}}
\ecvitem{}{}

\stepcounter{papercount}
\ecvitem{\thepapercount}{\begin{CJK}{UTF8}{min} 池田徹志、児堂義弘、Daniel REA、Francesco ZANLUNGO、塩見昌裕、神田崇行 \end{CJK}}
\ecvitem{}{\begin{CJK}{UTF8}{min} 街角における歩行者のサブゴール遷移モデル \end{CJK}}
\ecvitem{}{\begin{CJK}{UTF8}{min} 日本ロボット学会第31回学術講演会講演論文集RJS2013, 3I2-03, 2013 \end{CJK}}
\ecvitem{}{}

\stepcounter{papercount}
\ecvitem{\thepapercount}{\begin{CJK}{UTF8}{min} 塩見昌裕、Francesco ZANLUNGO、林宏太郎、神田崇行 \end{CJK}}
\ecvitem{}{\begin{CJK}{UTF8}{min} 街角で活動する移動ロボットのための歩行者シミュレータ\end{CJK}}
\ecvitem{}{\begin{CJK}{UTF8}{min} 日本ロボット学会第第30回学術講演会講演論文集RJS2012, 2N1-8, 2012 \end{CJK}}
\ecvitem{}{}

\stepcounter{papercount}
\ecvitem{\thepapercount}{\begin{CJK}{UTF8}{min} 塩見昌裕、Francesco ZANLUNGO、林宏太郎、神田崇行 \end{CJK}}
\ecvitem{}{\begin{CJK}{UTF8}{min} 歩行者モデルを用いた街角でのロボットナビゲーション \end{CJK}}
\ecvitem{}{\begin{CJK}{UTF8}{min} 日本ロボット学会第第30回学術講演会講演論文集RJS2012, 2N1-8, 2012 \end{CJK}}
\ecvitem{}{}

\ecvsection{Patents}

\ecvitem[10pt]{\large Registered patents}{}
\stepcounter{patentcount}
ecvitem{\thepatentcount}{T.~Ikeda, F.~Zanlungo, T. Miyashita, T.~Kanda} 
\ecvitem{}{\it System for the prediction of pedestrian motion and robot control}
\ecvitem{}{\it \begin{CJK}{UTF8}{min} (移動予測装置、ロボット制御装置、移動予測プログラムおよび移動予測方法) \end{CJK}}
\ecvitem{}{Japanese patent 5763384, registered on 19/6/2015}
\ecvitem{}{}

\stepcounter{patentcount}
\ecvitem{\thepatentcount}{M.~Shiomi, T.~Kanda, F.~Zanlungo, T. Ikeda} 
\ecvitem{}{\it A robot able to predict pedestrian motion and perform automatic collision avoidance}
\ecvitem{}{\it \begin{CJK}{UTF8}{min} (歩行者の軌跡を予測して自己の回避行動を決定するロボット) \end{CJK}}
\ecvitem{}{Japanese patent 5768273, registered on 3/7/2015}

\end{europecv}

\end{document}
